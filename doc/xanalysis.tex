\documentclass[a4paper,12pt]{article}
\usepackage{fontspec}
\usepackage{polyglossia}   %% загружает пакет многоязыковой вёрстки
\setdefaultlanguage{russian}  %% устанавливает главный язык документа
    %\setdefaultlanguage[babelshorthands=true]{russian}  %% вместо предыдущей строки; доступны команды из пакета babel для русского языка
\setotherlanguage{english} %% объявляет второй язык документа
\defaultfontfeatures{Ligatures={TeX},Renderer=Basic} %бывает нужно для того, чтобы -- превращалось в \endash, --- в \emdash, а << в «. У меня и так работает.
\setmainfont{Liberation Sans}     % задаёт \rmfamily, основной шрифт документа
\setsansfont{Liberation Sans}     % задаёт \sffamily, шрифт без засечек
\setmonofont{Liberation Mono} % задаёт \ttfamily, моношрифт
\usepackage{fancyhdr}
\usepackage{indentfirst}

%\renewcommand{\baselinestretch}{1.35}
\frenchspacing \sloppy \emergencystretch=5pt \textwidth=160mm \textheight=257mm
\oddsidemargin=4.6mm \headheight=10mm \footskip=10mm \headsep=10mm \topmargin=-1in \makeatletter

\renewcommand{\@oddfoot}{\ifnum\thepage>1{\hfil\large\thepage\hfil}\fi}
\renewcommand{\@evenfoot}{\ifnum\thepage>1{\hfil\large\thepage\hfil}\fi}
\renewcommand{\@oddhead}{}
\renewcommand{\@evenhead}{}

\newcommand{\No}{\textnumero}

\pagestyle{fancy} \fancyhf \lhead{} \chead{\ifnum\thepage>1{\hfil\large\thepage\hfil}\fi} \rhead{}
\renewcommand{\headrulewidth}{0pt}
\renewcommand{\footrulewidth}{0pt}

\begin{document}\large
\sffamily
\thispagestyle{fancy}

\begin{center}\LARGE\renewcommand{\baselinestretch}{.9}
\textbf{Анализ приложения {\tt baskerville}}\\
\end{center}

\section*{Введение}

Итак, разработано клиент-серверное приложение для защищенной пересылки файлов.
Файлы пересылаются односторонним образом: от клиента к серверу. Передаваемые файлы (их содержимое)
подписываются при помощи электронной подписи.
На сервере файл может быть размещен (по желанию клиента) в одной из \emph{корзин}.

\section{Атаки на приложение}

В этой секции мы опишем возможные векторы атаки на приложения и способы противодействия им.

\subsection{Конфиденциальность}

Данные передаются открыто и могут быть прочитаны нарушителем (задача защиты конфиденциальности в условии не ставилась).

\subsection{Подделка и нарушение целостности файлов при пересылке}

Для аутентификации файлов и контроля их целостности используется схема подписи RSA в сочетании с хэш-функцией SHA-512, реализация взята 
из OpenSSL. На данный момент нет причин считать, что у подписи RSA при использовании случайно выработанных секретных ключей длиной около 2048 битов
есть эксплуатируемые уязвимости, если секретный ключ не скомпрометирован.

\subsection{Открытые каталоги}

Режим листания работает без аутентификации, поэтому любой нарушитель может подключиться к серверу и ознакомиться с содержимым корзин.
Идентификаторы корзин дописываются к имени заданного каталога, при этом получается абсолютный путь до корзины. 
Теоретически, используя специально подобранные идентификаторы корзин (например, ``/../../../etc/''), нарушитель мог бы ознакомиться с содержимым всей файловой системы. В реализации, однако, обрабатываются только идентификаторы, прописанные в конфиге сервера, и атака невозможна.

\subsection{DoS-атака}

Благодаря тому, что запросы клиента не аутентифицируются, возможна DoS-атака путем посылки множества бессмысленных файлов, возможно, со случайно сгенерированными подписями, что заставит сервер тратить ресурсы на проверку неверных подписей.

\subsection{Replay-атака}

Нарушитель может повторно направить точную копию сообщения (вместе с ЦП), что приведет к перезаписи хранимого файла. Рассмотрим такую ситуацию:
\begin{itemize}
 \item Легальный клиент посылает файл 1.txt, нарушитель сохраняет копию его сообщения.
 \item Легальный клиент изменяет файл 1.txt и отправляет его на сервер.
 \item Нарушитель воспроизводит сохраненное сообщение, файл 1.txt перезаписывается старой копией.
\end{itemize}

\subsection{Атака с перезаписью файла} 

Нарушитель, перехвативший данные одного запроса, может послать его повторно, заменив  в запросе поле имени имя файла, и контент будет записан в файл с новым именем, возможно, перезаписав корректно размещенный файл. Для противодействия атаке нужно подписывать запрос на размещение файла.

Если имя файла содержит символы ``../'', то в принципе можно выйти за границы каталога и записать файл 
в произвольное место файловой системы. Чтобы этого не происходило, сервер обрезает имя файла при помощи \emph{basename()}.

 
\subsection{Атака с подменой корзины}

Нарушитель, перехвативший данные одного запроса, может послать его повторно, заменив id корзины, и файл будет записан не туда,
куда предполагал отправитель. Для противодействия атаке нужно подписывать запрос на размещение файла.

\subsection{Атака с подменой файла и корзины}

Комбинация предыдущих двух атак: нарушитель подменяет одновременно корзину и имя файла.

\section*{Выводы}

Против разработанного приложения можно применить целое семейство атак, связанных с тем, что проверяется лишь подпись под контентом файла.
Для борьбы с ними можно предложить следующие общие методы:
\begin{itemize}
\item Скорректировать процесс отправки, подписывая не только контент файла, но и его имя, и корзину назначения.
 \item Сделать протокол листания аутентифицированным, подписывая запрос. Еще лучше сделать весь протокол общения клиент-сервер трехшаговым:
 \begin{enumerate}
 \item Клиент обращается к серверу за ``квитанцией''.
  \item Сервер вырабатывает одноразовое случайное значение -- ``квитанцию'' Nonce, отправляет клиенту.
  \item Клиент готовит запрос, в который включает подпись от набора (квитанция, id корзины) для режима листания и
  (квитанция, id корзины, имя файла) для режима отправки, отдельно подписывая контент файла, и передает соответствующие поля и подпись (подписи) серверу. 
  \item Сервер обрабатывает только те запросы, для которых подпись верна. При этом уменьшится нагрузка на сервер при попытке DoS-атак.
 \end{enumerate}
\item Обернуть весь трафик протокола в TLS.
 \end{itemize}


\end{document}
